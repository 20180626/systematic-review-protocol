\documentclass[a4paper,12pt]{article}
\usepackage[T1]{fontenc}
\usepackage[utf8]{inputenc}
\usepackage{lmodern}
\usepackage{cite}
\usepackage{url}

\title{\textbf{Review Protocol: Investigating Cyberwarfare and 
    Compliance with International Humanitarian Law \\
    \small{Version: 1.0}}}
\author{\textbf{Lead Researcher:} SN 20180626} 
\date{\small{\today}}

\topmargin      0.0in
\headheight     0.0in
\headsep        0.0in
\oddsidemargin  0.0in
\evensidemargin 0.0in
\textheight     9.0in
\textwidth      6.5in

\pagestyle{plain}

\begin{document}

\maketitle

\section*{Summary}

\par Provide an abstract of your review topic. This should include 
what the policy problem is, the research gap, and a summary of the 
aims of the literature review (250 words)

\section*{Research Question}

\par Can International Humanitarian Law (IHL) sufficiently 
address cyberwarfare?

\subsection*{Sub-questions}

\begin{itemize}
    \item Can cyberweapons be adequately assessed by Article 36 
weapons reviews?
    \item Does offensive cyber action comply with Just War Theory?
    \item What are the negative externalities of cyberwarfare?
\end{itemize}

\section*{Academic Databases}

\par List all the databases you will be searching and a short 
description of what the scope and disciplinary focus of each database 
is.

\begin{itemize}
    \item ACM Digital Library
    \item IEEE Xplore
    \item Lecture Notes in Computer Science
    \item ProQuest Central
    \item ScienceDirect
    \item SCOPUS
    \item Web of Science
    \item JSTOR
    \item OECD iLibrary
    \item EBSCO
    \item EconPapers
\end{itemize}

\section*{Disciplines}

\begin{itemize}
    \item Social Sciences
    \item Computer Science
    \item Engineering
    \item Economics, Econometrics and Finance
    \item Decision Sciences
\end{itemize}

\section*{Inclusion and Exclusion}

\par The review will include:

\begin{itemize}
    \item All types of study design and methodologies (quantitative, 
qualitative, mixed-method, other literature reviews, case studies, 
case reports, perspectives and position papers)
    \item All studies must discuss online misinformation or 
disinformation (not other types of misinformation or disinformation)
    \item Etc...
\end{itemize}

\par The review will exclude: 

\begin{itemize}
    \item All articles that have not been published in peer-reviewed 
academic publications.
    \item Grey literature.
\end{itemize}

\section*{Search Strategy}

\par The review will use the following terms to search academic 
databases (e.g.):

\begin{itemize}
    \item (misinformation OR disinformation OR infodemic) AND (health* 
OR medical) AND (strateg* OR mitigate OR policy)
    \item Etc...
    \item Etc...
\end{itemize}

\par Searches will be limited to articles published between 
year-to-year. Provide one-two sentence justification for why you have 
chosen this period. 

\section*{Literature Management}

\par As I am writing in the LaTeX typesetting format, I am keeping a 
corresponding BibTeX database, using KBibTeX as my reference 
manager and Git for version control. I am hosting the .tex source for 
this document and the corresponding .bib database on a public GitHub 
repository, which can be found here: 
\url{https://github.com/20180626/systematic-review-protocol}.

\section*{Selection of Studies}

\par Provide a summary of your screening, inclusion and categorisation 
approach for the records retrieved from your search (i.e. articles, 
papers).

\section*{Strategy for Data Synthesis}

\par Provide a summary of the strategy for analysing selected records, 
for instance how you will summarise and analyse studies that respond 
to the research question and sub-questions, studies that use different 
methods, etc. 

\section*{Included Studies and Preliminary Results (700-900 words)}

\par There were approximately 10,000 unique results returned from 
systematic searches across the chosen academic databases. A total of 
3,010 results were prioritised for screening at title and abstract 
using machine learning, of which 306 met the inclusion criteria for 
full text screening. 80 of these articles met the criteria for 
inclusion in the review. Approximately 2,500 results were screened 
from website review and expert submission. 106 full texts were 
screened at full-text, of which 34 met the criteria for inclusion in 
the review. \\

\par 114 documents screened at full text met the criteria for 
inclusion in the REA. Due to the need for an efficient REA process, 
and reflecting the protocol, two reviewers manually prioritised 51 of 
the total of 114 documents meeting the criteria for inclusion, to 
determine which would be carried forward for data extraction and 
synthesis. Appendix 8 lists the remaining documents that met our 
inclusion criteria but were not synthesised and Appendix 6 lists 
evidence included for synthesis. \\

\par Tables of characteristics summarising the interventions, 
methodologies and outcomes from studies included for synthesis are 
provided in Appendix 7. \\

\textbf{Description of the included studies} \\

\par Due to the rapid nature of this review, we prioritised 51 of 114 
includable studies for synthesis. The analysis of the results 
presented below and the subsequent findings apply only to the 51 
studies included for synthesis. \\

\par Figure 1.2 indicates that the majority of studies included for 
synthesis evaluated infrastructure and road sign interventions (n=23), 
though it is important to note that this category is the broadest in 
terms of the number of different intervention types it includes. There 
were also substantive bodies of evidence for studies evaluating 
interventions covering law and rules of the road (n=15) and vehicle 
and equipment interventions (n=12). Fewer studies evaluated training 
and testing interventions (n=5) and road user education (n= 4).5. \\ 

\par In the 51 studies prioritised for synthesis we included 24 
evidence reviews and 27 quantitative primary studies. Of the evidence 
reviews nine described themselves as
systematic reviews, one as an REA and 14 as literature reviews. Of the 
quantitative primary studies, four are randomised controlled trials 
(RCTs), three employed differencein-differences, two are controlled 
before-after studies and seven are interrupted time series. Others 
include single group pre-test post-test (n=6), cross-sectional 
comparison studies (n=2) and a range of others including regression 
models (n=2), and multilevel models (n=1). \\

\par Typically the evidence reviews include evidence from multiple 
country settings (though two focussed on evidence from a single 
country). We also identified a limited number of European primary 
studies (n=12), of which three evaluated interventions in the UK. 
These were complemented by relevant primary studies from North America 
(n=14) and Australasia (n=1) identified through our search. \\

\par Regarding outcomes reported by synthesised studies, only five 
included studies explore both risk or perceived risk and 
participation. Thus, the evidence base is not well-suited to exploring 
whether participation in cycling and walking is affected by 
interventions designed to reduce risk or perceived risk. \\

\newpage

\begin{footnotesize}
    \bibliography{systematic-review-protocol}
    \bibliographystyle{unsrt}
\end{footnotesize}

\end{document}