%% LyX 2.3.6 created this file.  For more info, see http://www.lyx.org/.
%% Do not edit unless you really know what you are doing.
\documentclass[british]{article}
\usepackage[latin9]{inputenc}
\usepackage[a4paper]{geometry}
\geometry{verbose,tmargin=1in,bmargin=1in,lmargin=1in,rmargin=1in}
\setlength{\parskip}{\bigskipamount}
\setlength{\parindent}{0pt}
\usepackage{float}
\usepackage{booktabs}
\usepackage{url}
\usepackage{rotfloat}
\usepackage{setspace}
\onehalfspacing

\makeatletter

%%%%%%%%%%%%%%%%%%%%%%%%%%%%%% LyX specific LaTeX commands.
\DeclareTextSymbolDefault{\textquotedbl}{T1}
%% Because html converters don't know tabularnewline
\providecommand{\tabularnewline}{\\}

%%%%%%%%%%%%%%%%%%%%%%%%%%%%%% User specified LaTeX commands.
\usepackage{cite} 

\makeatother

\usepackage{babel}
\usepackage[style=trad-unsrt]{biblatex}
\addbibresource{systematic-review-protocol.bib}
\begin{document}
\title{\textbf{Review Protocol: Investigating Cyber--warfare and Compliance
with International Humanitarian Law (v1.0)}}
\date{\emph{21st February 2021}}
\author{\textbf{Lead Researcher: }SN 20180626}
\maketitle

\section{Summary}

The Fourth Geneva Convention was ratified in 1949 \cite{pictet1952geneva},
long before the advent of modern computers and the internet. As offensive
cyber action becomes increasingly prominent in the new \textquotedbl fifth
dimension\textquotedbl{} of warfare \cite{manson2011cyberwar}, with
rapid capacity--building for both offence and defence occurring in
many states \cite{manson2011cyberwar}, there is significant debate
between stakeholders across the public, private and third sectors
regarding whether Customary International Humanitarian Law (IHL) \cite{drmann2005customary}
is sufficient to address this new \textquotedbl wicked problem\textquotedbl{}
\cite{lin2012cyber}. 

This systematic review will be approached from a constructivist perspective,
with its purpose being to establish an understanding of the policy
issue of IHL compliance in cyber--warfare. The objective of this
review is to provide policymakers with a high--level overview of
the policy issue and efficacy of current interventions. To achieve
this objective, salient research literature will be systematically
identified, collected and critically assessed for inclusion in a qualitative
synthesis, from which answers may be inferred to the review's research
questions (See \textbf{SECTION 2}).

\section{Research Question}

Can Customary International Humanitarian Law (IHL) sufficiently address
cyber--warfare \cite{lin2012cyber}? 

\subsection{Sub--questions}
\begin{itemize}
\item Can cyber--attacks meet the principles of distinction, proportionality,
humanity and military necessity to justify use of force, as per Customary
IHL \cite{drmann2005customary}? 
\item Does offensive cyber action comply with Just War Theory \cite{eberle2013just}?
\item Can cyber--weapons be adequately assessed by the regulatory instrument
of Article 36 weapon reviews, as per Geneva Conventions Protocol I
\cite{international1987protocol}? 
\end{itemize}

\section{Academic Databases}

The review will collect studies from the following academic databases: 
\begin{itemize}
\item \textbf{ACM Digital Library} -- journal articles, conference proceedings,
magazines and other literature concerning computer science and information
and communication technologies. 
\item \textbf{IEEEXplore} -- journal articles, conference proceedings,
technical standards and other literature concerning computer science,
electrical engineering, electronics and other related fields.
\item \textbf{ProQuest }-- multi--disciplinary database including journal
articles, conference proceedings, trade publications and other literature. 
\item \textbf{Scopus} -- multi--disciplinary database including journal
articles, conference proceedings, trade publications, books and other
literature. 
\item \textbf{Web of Science} -- multi--disciplinary database including
journal articles, books and conference proceedings. 
\end{itemize}

\section{Disciplines}
\begin{itemize}
\item Political science
\item Military studies
\item Law
\item International relations
\item Peace and conflict studies
\item Technology
\item Public policy and administration
\end{itemize}

\section{Inclusion and Exclusion}

The review will include:
\begin{itemize}
\item All types of study design and methodologies, including but not limited
to, quantitative, qualitative and mixed--method reports, literature
reviews, surveys, case studies and technical reports. 
\item All articles concerning the issue of cyber--warfare conducted by
state, state--sponsored and non--state actors in the context of
IHL, including cyber--attacks, cyber--espionage, cyber--weapons
and cyber--deterrence. 
\end{itemize}
\newpage{}

The review will exclude:
\begin{itemize}
\item All articles that have not been published in peer--reviewed academic
journals. 
\item Grey literature, including trade journal and government publications.
\item All articles concerning the issues of information warfare \slash{}
disinformation \slash{} propaganda, hacktivism, lethal autonomous weapons
systems (LAWS), cyber--security, cyber--crime and cyber--terrorism.
\end{itemize}

\section{Search Strategy}

The review will use the following query strings containing salient
keywords to interrogate academic databases (See \textbf{SECTION 3}): 
\begin{itemize}
\item \texttt{(cyber AND war{*}) AND (law AND (\textquotedbl international
humanitarian\textquotedbl{} OR \textquotedbl armed conflict\textquotedbl )) }
\item \texttt{(cyber AND attack{*}) AND (principle{*} AND (distinction OR
necessity OR proportionality OR humanity)) }
\item \texttt{(cyber AND offens{*} AND action) AND (jus{*} AND (theory OR
tradition OR \textquotedbl ad bellum\textquotedbl{} OR \textquotedbl in
bello\textquotedbl )) }
\item \texttt{(cyber AND weapon{*}) AND (\textquotedbl article 36\textquotedbl{}
AND review)}
\end{itemize}
Searches will be limited to articles published between 2007--to--present.
2007 represented an epoch in cyber--warfare with significant cyber--attacks
suffered by Estonia \cite{haataja20172007}, allegedly perpetrated
by the Russian security apparatus \cite{haataja20172007}. These events
prompted the formation of the NATO Cooperative Cyber Defence Centre
of Excellence in Tallinn, Estonia \cite{burton2015nato}, and the
publication of the Tallinn Manual on the International Law Applicable
to Cyber Warfare \cite{schmitt2013tallinn} (See \textbf{SECTION 10.1}). 

\section{Literature Management}

As I am writing in the LaTeX typesetting format, I am managing references
for each stage of the systematic review in corresponding BibTeX databases,
using JabRef as my reference manager and Git for version control.
I am hosting the .tex source for this document and corresponding .bib
databases on a public GitHub repository (See \textbf{APPENDIX}).

\section{Selection of Studies}

I am using the PRISMA methodology for systematic reviews \cite{moher2010preferred},
beginning with identifying and enumerating salient studies collected
from selected academic databases (See \textbf{SECTION 3}), using query
strings containing salient keywords (See \textbf{SECTION 6}) to interrogate
the databases. After removing duplicate records, I will screen the
articles using the inclusion criteria (See \textbf{SECTION 5}) and
excluding those which do not meet it. Excluded articles will be kept
in a separate .bib database. Articles will then be assessed for eligibility,
with excluded articles kept in another .bib database (See \textbf{SECTION
10}). The included studies will be rigorously assessed in a qualitative
synthesis in the final review (See \textbf{SECTION 9}). 

\section{Strategy for Data Synthesis}

Articles included for qualitative synthesis will be coded to extract
salient information, such as research characteristics, data collection
methodologies and findings. This information will then be used to
map the research as a preparatory step for the data synthesis, potentially
providing context that may inform the interpretation of the included
articles. As this systematic review aims to establish an understanding
of the issue (See \textbf{SECTION 1}), I will utilise a configurative
approach to synthesise the included articles, employing inductive
reasoning to integrate and interpret them. This synthesis will aim
to answer the research questions (See \textbf{SECTION 2}) by identifying
commonalities and divergences in the qualitative research findings
of the included articles.

\section{Included Studies and Preliminary Results}

\begin{table}[H]
\begin{centering}
\begin{tabular}{lc}
\toprule 
\textbf{Stage} & \textbf{Total n}\tabularnewline
\midrule
Database search results & 5861\tabularnewline
Duplicates removed & 4690\tabularnewline
Screened (title-abs-key) & 852\tabularnewline
Excluded (title-abs-key) & 3838\tabularnewline
Assessed (full page) & 62\tabularnewline
Excluded (full page) & 770\tabularnewline
Included (qualitative) & 49\tabularnewline
\bottomrule
\end{tabular}
\par\end{centering}
\textbf{\caption{PRISMA Flow}
}
\end{table}

Search results from the selected academic databases (See \textbf{SECTION
3}), interrogated using query strings (See \textbf{SECTION 6}) and
database APIs where possible, returned 5861 results. After cleaning
the reference database and removing duplicates, 4690 articles remained
and were prioritised for screening at the title, abstract and keyword
level (title-abs-key). To automate the screening process, I leveraged
machine learning by installing and deploying the open--source Active
learning for Systematic Reviews (ASReview) software (See \textbf{APPENDIX})
on a dedicated cloud--based virtual machine. Of the 4690 articles
prioritised for screening, 852 met the criteria for full page assessment
(See \textbf{SECTION 5}), with 3838 articles excluded. To automate
the process for full page assessment, I again leveraged machine learning
by cloning and deploying the open--source General Architecture for
Text Engineering (GATE) GitHub repository (See \textbf{APPENDIX})
on another dedicated cloud--based virtual machine. Of the 852 articles
prioritised for full page assessment, 62 met the criteria for inclusion
in the evidence synthesis (See \textbf{SECTION 5}), with 770 articles
excluded. Of the remaining 62 articles, 49 were prioritised for inclusion
in the qualitative synthesis. 

\subsection{Description of Included Studies for Qualitative Synthesis}

\begin{sidewaystable}[H]
\begin{centering}
\begin{tabular}{lcclccccclc}
\cmidrule{1-2} \cmidrule{2-2} \cmidrule{4-5} \cmidrule{5-5} \cmidrule{7-8} \cmidrule{8-8} \cmidrule{10-11} \cmidrule{11-11} 
\textbf{Country} & \textbf{Total n} &  & \textbf{Keyword} & \textbf{Total n} &  & \textbf{Year} & \textbf{Total n} &  & \textbf{Subject} & \textbf{Total n}\tabularnewline
\cmidrule{1-2} \cmidrule{2-2} \cmidrule{4-5} \cmidrule{5-5} \cmidrule{7-8} \cmidrule{8-8} \cmidrule{10-11} \cmidrule{11-11} 
United States & 18 &  & Cyberspace & 5 &  & 2020 & 4 &  & Social Sciences & 43\tabularnewline
United Kingdom & 15 &  & Tallinn Manual & 5 &  & 2019 & 5 &  & Engineering & 12\tabularnewline
Australia & 3 &  & Cyber & 4 &  & 2018 & 4 &  & Arts and Humanities & 10\tabularnewline
Canada & 3 &  & Cyber Operations & 4 &  & 2017 & 4 &  & Computer Science & 10\tabularnewline
Netherlands & 3 &  & Attribution & 3 &  & 2016 & 6 &  & Business & 2\tabularnewline
South Korea & 2 &  & Cyber Attack & 3 &  & 2015 & 8 &  & Decision Sciences & 2\tabularnewline
Austria & 1 &  & Cyber Security & 3 &  & 2014 & 0 &  & Psychology & 2\tabularnewline
Belgium & 1 &  & Cyber War & 3 &  & 2013 & 7 &  & Economics & 1\tabularnewline
Denmark & 1 &  & Ethics & 3 &  & 2012 & 7 &  & Environmental Science & 1\tabularnewline
Germany & 1 &  & International Humanitarian Law & 3 &  & 2011 & 2 &  &  & \tabularnewline
Greece & 1 &  & International Law & 3 &  & 2010 & 2 &  &  & \tabularnewline
Israel & 1 &  & Jus Ad Bellum & 3 &  &  &  &  &  & \tabularnewline
Japan & 1 &  &  &  &  &  &  &  &  & \tabularnewline
Norway & 1 &  &  &  &  &  &  &  &  & \tabularnewline
Sweden & 1 &  &  &  &  &  &  &  &  & \tabularnewline
Thailand & 1 &  &  &  &  &  &  &  &  & \tabularnewline
Undefined & 3 &  &  &  &  &  &  &  &  & \tabularnewline
\cmidrule{1-2} \cmidrule{2-2} \cmidrule{4-5} \cmidrule{5-5} \cmidrule{7-8} \cmidrule{8-8} \cmidrule{10-11} \cmidrule{11-11} 
\end{tabular}
\par\end{centering}
\textbf{\caption{\textbf{Articles included for qualitative synthesis by country, keyword,
year and subject}}
}
\end{sidewaystable}

The 49 articles prioritised for qualitative synthesis comprised 46
research articles and three reviews. The following analysis applies
only to these articles. The majority of articles included for qualitative
synthesis were published in NATO member states (See \textbf{TABLE
2}), with publications from the United States and United Kingdom featuring
prominently in the results. These are complimented by publications
from Australia and Canada, which may infer a positive correlation
between member states of the \textquotedbl Five Eyes\textquotedbl{}
signals intelligence (SIGINT) alliance \cite{pfluke2019history} and
policy concerns regarding cyber--warfare. This inference may be supported
by the author affiliation of eight of the included articles with the
defence apparatus of three of the \textquotedbl Five Eyes\textquotedbl{}
member states\footnote{\textbf{Author's note:} funding sponsorship status is unclear.}.

Analysing the included articles by subject area, the majority of studies
concern the social sciences, complimented by engineering, arts and
humanities and computer science (See \textbf{TABLE 2}). Analysing
the included articles by year, there are two deltas which bear scrutiny
(See \textbf{TABLE 2}). The first occurs in 2012, where the number
of publications more than tripled from the previous year, which may
be correlated \textit{a priori} with the unprecedented Stuxnet cyber--attack
against Iran's nuclear program in 2010 \cite{langner2011stuxnet}
and the public release of the United States Department of Defense's
2010 Annual Report to Congress concerning Military and Security Developments
Involving the People's Republic of China \cite{gates2010military},
which warned for the first time of the Chinese People's Liberation
Army's capacity--building for offensive cyber operations. The second
delta occurs in 2015 where the number of publications increased by
eight from the previous year, which may be correlated \textit{a priori}
with the 2014 Sony Pictures Entertainment Hack \cite{haggard2015north},
a prominent confidentiality--related cyber--attack formally attributed
by the United States Federal Bureau of Investigation and Department
of Justice to threat actors affiliated with the North Korean government's
intelligence apparatus\footnote{\textbf{Author's note:} this attribution has been publicly disputed
by members of the cyber--security community.}.

Keyword analysis of the included articles (See \textbf{TABLE 2}) includes
prominent keywords that appeared in the keyword section of more than
two articles. While many are expected, such as \textquotedbl cyber
war\textquotedbl{} and \textquotedbl cyber attack\textquotedbl , two
keywords offer potential insight regarding the context of the included
articles, which may inform the data synthesis. The first, \textquotedbl attribution\textquotedbl ,
included in the keyword section of three of the included articles
for synthesis, concerns the policy issue of attributing offensive
cyber operations to actors \cite{dipert2010ethics}. Covert cyber
operations, or state actors leveraging proxies to conduct offensive
cyber operations, may contravene Rule 149 of Customary IHL \cite{drmann2005customary},
regarding the responsibility of a state for violations of IHL committed
directly or via proxy. This is particularly salient given the acknowledged
connections between several state SIGINT apparatuses and cyber threat
actors. An exemplar of this issue is the identified connection between
the United States National Security Agency's elite Tailored Access
Operations unit \cite{loleski2019cold} and the Equation Group \cite{van2017impact},
a cyber threat actor group classified as an advanced persistent threat
\cite{alshamrani2019survey} and linked to cyber--attacks against
Iran, Russia and Syria among others, allegedly including the Stuxnet
cyber--attack \cite{langner2011stuxnet}. 

The second prominent keyword, \textquotedbl Tallinn Manual\textquotedbl ,
included in the keyword section of five of the included articles for
synthesis, refers to the Tallinn Manual on the International Law Applicable
to Cyber Warfare \cite{schmitt2013tallinn}. Initially published by
the NATO Cooperative Cyber Defence Centre of Excellence in 2013\footnote{\textbf{Author's note:} a second edition of the Tallinn Manual was
published in 2017 \cite{schmitt2017tallinn}.}, the Tallinn manual is a non--binding academic study concerning
the governance of cyber--warfare under IHL, authored by a broad church
of representatives from academia, civil society and government. Given
its seminal status and influence on policy, which has been subject
to rigorous scrutiny by legal scholars and practitioners \cite{kessler2013expertise,fleck2013searching,von2012tallinn},
it will undoubtedly inform the synthesis of this review.\newpage{}

\section*{Appendix}

1. BibTeX databases for each stage of the PRISMA workflow outlined
in this protocol (See \textbf{SECTION 10}), including articles included
for qualitative synthesis, may be found at: \url{https://github.com/20180626/systematic-review-protocol}.

2. Active learning for Systematic Reviews (ASReview) 
\begin{itemize}
\item Source code: \url{https://github.com/asreview/asreview} 
\item Documentation: \url{https://asreview.readthedocs.io} 
\item White paper: \fullcite{van2020asreview}.
\end{itemize}
3. General Architecture for Text Engineering (GATE) 
\begin{itemize}
\item Source code: \url{https://github.com/GateNLP/gate-core} 
\item Documentation: \fullcite{cunningham2009developing}. 
\item White paper: \fullcite{cunningham2002gate}.
\end{itemize}
\newpage{}

\printbibliography

\end{document}
